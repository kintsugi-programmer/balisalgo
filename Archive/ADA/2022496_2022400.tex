\documentclass{article}
\usepackage[utf8]{inputenc}
\usepackage{amsmath}
\usepackage{algorithm}
\usepackage{algpseudocode}
\usepackage[margin=2.2cm]{geometry}

\title{ADA-2024: Homework-5}
\author{Siddhant Bali (2022496) \and Rijusmit Biswas (2022400)}
\date{April 21, 2024}

\begin{document}
\maketitle

\section{Problem Statement}
Suppose you are given a set of boxes, each specified by their height, width, and depth in centimeters. All three side lengths of every box are strictly between 10 cm and 20 cm. One box can be placed in another if the first box can be rotated so that its height, width, and depth are respectively smaller than the height, width, and depth of the second box. Boxes can be nested recursively. Design an algorithm to nest the boxes so that the number of visible boxes is as small as possible.

\section{Algorithm Description}
To solve this problem, we'll construct a flow network where vertices represent boxes and edges represent nesting relationships. We'll then apply the Ford-Fulkerson algorithm using Breadth First Search at Queue implementation with relevance to find the maximum flow, which corresponds to the minimum number of visible boxes.

\section{Recurrence Relation}
Not applicable for this problem.

\section{Complexity Analysis}

\begin{itemize}
    \item Time Complexity :
    
    Building the flow network takes $\mathcal{O}(n^2)$ time, where $n$ is the number of boxes. Using the Ford-Fulkerson algorithm to find the maximum flow adds $\mathcal{O}(f \cdot (|V| + |E|))$ complexity, where $f$ is the maximum flow, and $|V|$ and $|E|$ are the number of vertices and edges, respectively. So, overall, it's $\mathcal{O}(n^2 + f \cdot (|V| + |E|))$.

\end{itemize}

\section{Pseudocode}
\begin{algorithm}
\caption{NestBoxes}
\begin{algorithmic}[1]
\Function{NestBoxes}{$boxes$}
    \State $n \gets \text{length of } boxes$
    \State Initialize $capacity$ matrix with all zeros
    \State $source \gets 0$
    \State $sink \gets 2 \times n + 1$
    \For{$i \gets 1$ \textbf{to} $n$}
        \State $capacity[source][i] \gets 1$
        \State $capacity[i][i + n] \gets 1$
        \For{$j \gets 1$ \textbf{to} $n$}
            \If{$i \neq j$ \textbf{and} \Call{canBeNested}{$boxes[i-1]$, $boxes[j-1]$}}
                \State $capacity[i + n][j] \gets 1$
            \EndIf
        \EndFor
    \EndFor
    \State $maxFlow \gets$ \Call{FordFulkerson}{$capacity$, $source$, $sink$, $2 \times n + 2$}
    \State \Return $n - \text{maxFlow}$
\EndFunction
\end{algorithmic}
\end{algorithm}

\begin{algorithm}
\caption{canBeNested}
\begin{algorithmic}[1]
\Function{canBeNested}{$A$, $B$}
    \State \Return $(A.height < B.height$ \textbf{and} $A.width < B.width$ \textbf{and} $A.depth < B.depth)$
\EndFunction
\end{algorithmic}
\end{algorithm}

\begin{algorithm}
\caption{FordFulkerson}
\begin{algorithmic}[1]
\Function{FordFulkerson}{$capacity$, $source$, $sink$, $n$}
    \State $maxFlow \gets 0$
    \State Initialize $parent$ array
    \While{\Call{bfs}{$capacity$, $source$, $sink$, $n$, $parent$}}
        \State $pathFlow \gets \infty$
        \For{$v \gets sink$ \textbf{to} $source$}
            \State $u \gets parent[v]$
            \State $pathFlow \gets \min(pathFlow, capacity[u][v])$
        \EndFor
        \For{$v \gets sink$ \textbf{to} $source$}
            \State $u \gets parent[v]$
            \State $capacity[u][v] \gets capacity[u][v] - pathFlow$
            \State $capacity[v][u] \gets capacity[v][u] + pathFlow$
        \EndFor
        \State $maxFlow \gets maxFlow + pathFlow$
    \EndWhile
    \State \Return $maxFlow$
\EndFunction
\end{algorithmic}
\end{algorithm}

\begin{algorithm}
\caption{bfs}
\begin{algorithmic}[1]
\Function{bfs}{$capacity$, $source$, $sink$, $n$, $parent$}
    \State Initialize $visited$ array with $false$ values
    \State Create an empty queue
    \State Enqueue $source$ into the queue
    \State $visited[source] \gets true$
    \State $parent[source] \gets -1$

    \While{queue is not empty}
        \State $u \gets$ Dequeue from the queue
        \For{$v \gets 0$ \textbf{to} $n-1$}
            \If{not $visited[v]$ \textbf{and} $capacity[u][v] > 0$}
                \State Enqueue $v$ into the queue
                \State $parent[v] \gets u$
                \State $visited[v] \gets true$
            \EndIf
        \EndFor
    \EndWhile
    \State \Return $visited[sink]$
\EndFunction
\end{algorithmic}
\end{algorithm}


\section{Proof of Correctness}
We construct a flow network where each vertex represents a box, and there is an edge from vertex $i$ to vertex $j$ if box $i$ can be nested inside box $j$. We also add a source vertex $s$ with edges to all box vertices, and a sink vertex $t$ with edges from all box vertices.

The maximum flow in this network represents the maximum number of boxes that can be nested. This is because each unit of flow corresponds to a box being nested inside another box.

Therefore, the minimum number of visible boxes is equal to the total number of boxes minus the maximum flow value. This is because the boxes that are nested inside other boxes are not visible.

By using the Ford-Fulkerson algorithm to find the maximum flow in the constructed network, we can correctly compute the minimum number of visible boxes for the given set of boxes.
\end{document}
